%%%%%%%%%%%%%%%%%%%%%%%%%%%%%%%%%%%%%%%%%
%	PACKAGES AND OTHER DOCUMENT CONFIGURATIONS
%----------------------------------------------------------------------------------------
\documentclass{report}
\usepackage[a4paper]{geometry} % Use A4 paper margins
\usepackage{color} % Required for color customization
\usepackage{times}
\usepackage{graphicx}
\usepackage{amsmath}
\usepackage[numbers,sort&compress]{natbib}
\definecolor{Navy}{RGB}{50,90,122} % Define a custom color for the heading box
%----------------------------------------------------------------------------------------
\begin{document}
% ----------------------------------------------------------------------------------------

As I have been slowly learning about what we are doing through
imperfectly communicating with Eric and Tom I have slowly been
figuring out what this project is and what the goals of it are.



The first model that I could understand is the non-time
dependent model put forth in Equation 1.  This model was coded up in
the first week that I joined the project and was shared with the
team. It allows for the input of 

























Just to keep it simple to implement I just write the model as a
matrix such that $A_{t+1}=T*A_t$.  Note that currently $F$ is just the
number of eggs, but it could easily be some sort of density dependent
function. The nice thing about doing things this way is that simply
replacing the matrix $T$ with a new matrix effectively changes the
model without much code rewriting. I made a version with repeat
spawners and without repeat spawners. 

I spent some time trying to fit to the age data. However, there are a
lot of parameters. Initial abundance in each class, freshwater
survival, marine survival, the proportion of age 4 and 5 fish that
spawn.  Many of these parameters get effectively multiplied by each
other to predict age 4,5,6 fish. It did not seem to work that well.
I will spend some more time trying to calibrate after the meeting
tomorrow.  Certainly marine survival could change overtime, but we
need a likelihood function to maximize. 


I am currently not keeping direct
count of age 7+ fish, but 6+ year old fish always spawn and mostly
die. The spreadsheet says there are 6 percent age 6 fish and almost no
age 7 fish.  However, it is certainly possible to add more age
classes.

I apologize for not knowing how to cut and paste matrices between word
and R. I am trying to develop my word skills, but I also wanted to
make sure that the matrices in this document were exactly as in the R
code. 

\section{No Repeat Spawners}
This is shown by the simple model under ``Are there Repeat
Spawners?''.   

\[
 T = 
  \begin{matrix}
                  0&0&0&F&0&F&0&F\\
                  SF&0&0&0&0&0&0&0\\
                  0&SF&0&0&0&0&0&0\\
                  0&0&a*SP&0&0&0&0&0\\
                  0&0&(1-a)*SP&0&0&0&0&0\\
                  0&0&0&0& b*SM&0&0&0\\
                  0&0&0&0& (1-b)*SM&0&0&0\\
                  0&0&0&0&0&0&SM&SM

  \end{matrix}
\]

Or equivalently

\begin{align*} 
  A1_{T+1}&=F \cdot A4_Ts+F \cdot A5s_T+F \cdot A6_T\\
  A2_{T+1}&=SF \cdot A1_T\\
  A3_{T+1}&=SF \cdot A2_T\\
  A4s_{T+1}&=aSP \cdot A3_T\\
  A4o_{T+1}&=(1-a)SP \cdot A3_T\\
  A5s_{T+1}&=bSM \cdot A4_To\\
  A5o_{T+1}&=(1-b)SM \cdot A4_To\\
  A6_{T+1}&=SM \cdot A5o_T+SM \cdot A6_T\\
\end{align*}

    % T<-t(matrix(c(0,0,0,F,0,F,0,F, #1
    %              SF,0,0,0,0,0,0,0,  #2
    %              0,SP,0,0,0,0,0,0,  #3
    %              0,0,a*SM,0,0,0,0,0,  #4s
    %              0,0,(1-a)*SM,0,SM,0,0,0,  #40
    %              0,0,0,SM, 0,b*SM,0,0,  #5
    %              0,0,0,0, 0,(1-b)*SM,SM,0,  #5
    %              0,0,0,0,0, 0,0,SM),#6
    %             nrow=8))



\section{Repeat Spawners}



\[
 T = 
  \begin{matrix}
                  0&0&0&F&0&F&0&F\\
                  SF&0&0&0&0&0&0&0\\
                  0&SF&0&0&0&0&0&0\\
                  0&0&a*SP&0&0&0&0&0\\
                  0&0&(1-a)*SP&0&0&0&0&0\\
                  0&0&0&R*SM& b*SM&0&0&0\\
                  0&0&0&0& (1-b)*SM&0&0&0\\
                  0&0&0&0&0&R*SM&SM&SM

  \end{matrix}
\]




\end{document}

